%Two resources useful for abstract writing.
% Guidance of how to write an abstract/summary provided by Nature: https://cbs.umn.edu/sites/cbs.umn.edu/files/public/downloads/Annotated_Nature_abstract.pdf %https://writingcenter.gmu.edu/guides/writing-an-abstract
\chapter*{\center \Large  Abstract}
%%%%%%%%%%%%%%%%%%%%%%%%%%%%%%%%%%%
% Replace all text with your text
%%%%%%%%%%%%%%%%%%%%%%%%%%%%%%%%%%%

This workshop presents a comprehensive systems analysis of the CIBMTR (Center for International Blood and Marrow Transplant Research) Kaggle competition focused on equity in post-hematopoietic cell transplant (HCT) survival predictions. The competition challenges participants to develop machine learning models that accurately predict survival outcomes following allogeneic hematopoietic cell transplantation while ensuring fairness across different racial and socioeconomic groups.

\vspace{0.3cm}

Our analysis examines the complexity of the medical system surrounding HCT procedures, identifies key variables affecting patient outcomes, and evaluates the dual requirements of accuracy and equity in predictive modeling. The study reveals the intricate relationships between clinical, genetic, and demographic factors that influence post-transplant survival, highlighting the system's sensitivity to small parameter variations and the presence of chaotic behavior in medical outcomes.

\vspace{0.3cm}

Key findings include the identification of critical sensitivity parameters such as patient age, disease risk indices, genetic compatibility scores, and comorbidities. The analysis emphasizes the importance of the stratified C-index evaluation metric, which ensures model performance is consistent across ethnic subgroups. This work contributes to understanding how systems thinking can be applied to improve both the accuracy and fairness of medical prediction models, with direct implications for personalized medicine and transplant care.

\vspace{0.3cm}

The integration of recent advances in machine learning approaches for hematopoietic stem cell research demonstrates the evolving landscape of predictive modeling in healthcare. Our recommendations include implementing ensemble methods, maintaining demographic balance in cross-validation strategies, and incorporating domain expertise in feature engineering to address the inherent complexity and randomness in biological systems.

\vspace{0.3cm}

\textbf{Keywords:} Hematopoietic Cell Transplantation, Machine Learning, Healthcare Equity, Systems Analysis, Survival Prediction, Medical Informatics, Artificial Intelligence, Predictive Modeling